\documentclass[a4paper,12pt]{article}
\usepackage[ngerman]{babel}
\usepackage[T1]{fontenc}
\usepackage[utf8]{inputenc}
\usepackage{lmodern}
\usepackage[margin=3cm]{geometry}
\usepackage{setspace}
\usepackage{amsmath, amssymb}
\usepackage{graphicx}
\usepackage{caption}
\usepackage{subcaption}
\usepackage[colorlinks=true, allcolors=gray]{hyperref}
\usepackage{csquotes}
\usepackage{biblatex}
\usepackage{fancyhdr}
\usepackage{tocbibind}


\title{Bachelorarbeit}
\author{Tom Becke}
\date{\today}
\setlength{\parindent}{0pt}

\begin{document}

\doublespacing

\maketitle

\begin{center}
    {\LARGE \textbf{Leistungs- und Renderzeitvergleich von React Native und Flutter: Eine Analyse der Performance moderner Cross-Plattform-Frameworks}}
\end{center}

\begin{abstract}
    Abstract kommt hier irgendwann hin.
\end{abstract}

\newpage

\tableofcontents

\newpage

\section{Einführung}
Heutzutage gibt es verschiedene Methoden, um Software effizient zu entwickeln. Um eine gelungene Oberfläche für den Benutzer zu gestalten, gibt es verschiedene Ansätze. Von der nativen Entwicklung, die explizit für das jeweilige Gerät optimiert ist, über die Cross-Platform Entwicklung, die einen Code verwendet um diesen auf verschiedenen Geräten abzubilden, bis hin zur webbasierten Entwicklung, die universell über einen Browser auf allen Geräten funktioniert, gibt es für den Entwickler diverse Möglichkeiten. Dabei sind viele Kriterien von großer Bedeutung. Dazu zählt die Auslastung auf den Endgeräten, da diese das Erlebnis, welches der Nutzer mit der Software hat, stark beeinflusst. Hinzu kommt die Performance, wie schnell die gewünschten Komponenten der Software angezeigt werden und in welcher Bildwiederholrate dies geschieht. Diese Faktoren werden in dieser Bachelorarbeit näher beleuchtet und in einer Studie verglichen.

Cross-Platform Frameworks haben in letzter Zeit viel Aufmerksamkeit erhalten, da sie in der Praxis wenig zeitintensiv in der Handhabung sind und bekannte Methoden durch verschiedene Techniken auf unterschiedliche Geräte projizieren können.

Die am häufigsten verwendeten Cross-Plattform Frameworks Flutter und React Native stehen im Mittelpunkt dieser Arbeit. Diese beiden Frameworks werden hinsichtlich der Auslastung auf den Endgeräten sowie der Performance beim Rendern verschiedener Komponenten verglichen. Die Forschungsfrage entstand vor dem Hintergrund, welches der beiden Frameworks in der Praxis für einen Großteil der Endgeräte in Bezug auf die genannten Aspekte optimaler ist. In einem Versuchsaufbau mit identischen Voraussetzungen werden die Frameworks auf verschiedene Punkte getestet und die Ergebnisse verglichen. Die abschließende Diskussion gibt Aufschluss darüber, welches der beiden Frameworks unter Berücksichtigung der genannten Punkte in der Praxis besser geeignet ist.

\section{Methoden der Softwareentwicklung}
Die Softwareentwicklung lässt sich in verschiedene Kategorien einteilen. Die Backend-Entwicklung befasst sich mit der Serverseite einer Anwendung. Es werden Anfragen im HTTP-Format an das Backend geschickt, das die Anfrage verarbeitet und die Antwort zurückliefert. Datenbanken, Microservices und weitere diverse Funktionalitäten werden im Backend umgesetzt und sind dadurch strikt vom Frontend getrennt.

\vspace{0.5cm}

Das Frontend beschreibt den Teil einer Anwendung, der auf dem Endgerät läuft und dafür zuständig ist, Eingaben zu verwalten oder Daten vom Backend darzustellen. Die Benutzeroberfläche wird durch verschiedene Elemente repräsentiert, die der Anzeige von Daten dienen oder dem Benutzer die Möglichkeit bieten, mit der Schnittstelle zu interagieren. In dieser Arbeit steht das Frontend im Vordergrund, da der Fokus auf diesem Bereich liegt.

\vspace{0.5cm}

### Maschinencode

Die tiefste Ebene der Programmiersprachen bilden die Maschinensprachen oder Assembler-Sprachen ab. Diese Sprachen werden direkt auf dem Prozessor des Zielsystems interpretiert und ausgeführt. Da verschiedene Prozessoren unterschiedliche Befehle verarbeiten können, muss der Code für das jeweilige Zielsystem ausgelegt sein. Assembler-Sprachen sind also plattformspezifisch und erfordern eine genaue Abstimmung auf das Endgerät.

\vspace{0.5cm}

### Bytecode

Einige Programmiersprachen verwenden Bytecode, um den geschriebenen Quellcode in eine Zwischensprache zu übersetzen. Dies ermöglicht es, den Code auf verschiedenen Endgeräten laufen zu lassen, ohne ihn anpassen zu müssen. Sprachen wie Java nutzen Bytecode und benötigen eine virtuelle Maschine (JVM), die den Zwischencode zur Laufzeit in Maschinensprache für das jeweilige Zielsystem übersetzt. Ein Vorteil der Bytecode-Verwendung ist die Plattformunabhängigkeit.

\vspace{0.5cm}

### Native Frontend-Entwicklung - iOS (Swift) und Android (Kotlin)

Möchte man auf bestimmte Funktionen eines Endgeräts zugreifen, ist eine native Entwicklung erforderlich. Bei Android-Geräten wird häufig Kotlin verwendet, während iOS-Apps in Swift programmiert werden. Native Software greift direkt auf das Betriebssystem zu und muss daher plattformspezifisch sein. Kotlin und Swift bieten leistungsstarke Möglichkeiten, um direkt auf die Hardware und nativen Funktionen der Geräte zuzugreifen.

\vspace{0.5cm}

### Webbasierte Software

Die fundamentalste Möglichkeit, ein Frontend zu entwickeln, ist die webbasierte Programmierung. Mit HTML (Hyper Text Markup Language) werden Komponenten entwickelt, die von einem Webbrowser interpretiert und dargestellt werden. HTML-Dateien beinhalten jedoch keine Logik. Um die Benutzeroberfläche mit Funktionen auszustatten, wird JavaScript genutzt, das dynamisch zur Laufzeit kompiliert wird. Um die Benutzeroberfläche zu manipulieren, wird das DOM (Document Object Model) verwendet, das eine Schnittstelle zwischen HTML und JavaScript darstellt.

\vspace{0.5cm}

Frameworks wie Angular, React oder Vue.js erleichtern die Entwicklung von Webanwendungen durch strukturierte Ansätze und die Möglichkeit, wiederverwendbare Komponenten zu erstellen. Diese Frameworks nutzen oft ein virtuelles DOM, das für eine effiziente Aktualisierung und Verwaltung der Benutzeroberfläche sorgt.

\vspace{0.5cm}

### Vorteile webbasierter Software

- Funktioniert in jedem Webbrowser auf jedem Endgerät \\
- Breit gefächertes Expertenwissen \\
- Viele Frameworks und Bibliotheken erleichtern die Arbeit

\vspace{0.5cm}

### Nachteile webbasierter Software

- Kompilierung nur durch den Webbrowser \\
- Fehlende Möglichkeit, spezifische Funktionen des Endgeräts zu nutzen \\
- Geringere Performance \\
- Fehlende Sicherheit durch sichtbaren Code

\vspace{0.5cm}

### Cross-Platform Entwicklung - React Native oder Flutter

Cross-Platform-Entwicklung ermöglicht es, einen Quellcode zu schreiben, der auf verschiedenen Endgeräten läuft. Dies spart Zeit und Kosten, da nicht für jedes Betriebssystem ein separater Code entwickelt werden muss. Frameworks wie React Native und Flutter erlauben es, den Quellcode in die native Sprache des Zielsystems zu übersetzen, was die Leistung verbessert und den Zugriff auf native Funktionen ermöglicht.

\vspace{0.5cm}

### Vorteile der Cross-Platform-Entwicklung

- Zeitersparnis \\
- Wiederverwendbarkeit des Codes \\
- Ähnliches Design wie bei nativen Komponenten \\
- Leichtere Instandhaltung und Einbindung neuer Updates \\
- Höhere Anzahl an Entwicklern

\vspace{0.5cm}

### Nachteile der Cross-Platform-Entwicklung

- Vereinzelt fehlende native Funktionen \\
- Größere kompilierte Apps auf den Endgeräten \\
- Potenzielle Einbußen in der Leistung

\vspace{0.5cm}

Aufgrund der steigenden Popularität von Cross-Plattform-Frameworks und der Vielzahl an Vorteilen werden im Folgenden zwei der bekanntesten Frameworks, Flutter und React Native, näher betrachtet.


\section{React Native im Detail}
React Native wurde von Meta Platforms Inc. entwickelt und erschien im Jahr 2015. Die Handhabung von React Native ist sehr ähnlich zu der von klassischem React für die webbasierte Softwareentwicklung. Der komponentenbasierte Ansatz beim Entwickeln und die Möglichkeit, JavaScript oder TypeScript zu verwenden, bieten erfahrenen Entwicklern eine vertraute Umgebung. 

\vspace{0.5cm}

Der Hauptunterschied zu klassischem React liegt darin, dass nicht der DOM über einen virtuellen DOM manipuliert wird, sondern der JavaScript-Code direkt auf dem Endgerät kompiliert wird.

\vspace{0.5cm}

Die Kompilierung des JavaScript-Codes zum nativen Code erfolgt über sogenannte Brücken, über die die Übersetzung stattfindet. Über diese bidirektionale Verbindung werden Nachrichten gesendet, die angeben, welche Aktionen ausgeführt werden sollen und wie dies geschehen soll. Beispielsweise wird von der JavaScript-Seite eine Information gesendet, dass eine \textit{View} dargestellt werden soll, die von der nativen Seite aus aufgebaut werden muss. 

\vspace{0.5cm}

Umgekehrt funktioniert die Kommunikation ebenso: Wenn eine Eingabe in der Ansicht erfolgt, wird eine Nachricht über die Brücke an die JavaScript-Seite gesendet. Diese Informationen werden asynchron übertragen und von der Brücke zwischen den beiden Parteien verwaltet. Die Asynchronität ermöglicht eine reibungslose Darstellung der Ansichten, wodurch das Ziel einer flüssigen Darstellung von etwa 60 Bildern pro Sekunde erreicht wird. Diese Architektur lässt sich durch ihr entkoppeltes System und die Flexibilität für die Integration mit verschiedenen Frameworks und Systemen stetig erweitern.

\vspace{0.5cm}

Eine zu hohe Anzahl an Nachrichten, die über die Brücke gesendet werden, kann zu einem Stau führen, was Leistungseinbußen zur Folge hat. \textit{Hermes} ist eine JavaScript-Engine, die genutzt wird, um den Quellcode direkt auf dem Endgerät auszuführen. Sie führt Bytecode direkt auf dem Endgerät aus, was zu einer besseren Performance führt. Falls Hermes deaktiviert ist, wird \textit{JavaScriptCore} verwendet, das auf iOS-Geräten den Nachteil der fehlenden \textit{JIT}-Kompilierung (\textit{Just-in-Time}) aufweist. 

\vspace{0.5cm}

Ein weiterer wichtiger Bestandteil des Kompilierungs- und Bauprozesses ist der \textit{Metro-Bundler}, der die JavaScript-Dateien bündelt und für die Ausführung auf dem jeweiligen Endgerät vorbereitet.

\vspace{0.5cm}

React Native verwendet native Build-Systeme, um die Applikationen für die jeweiligen Endgeräte zu bauen. Bei iOS und Android sind die Schritte nahezu identisch: Android nutzt \textit{Gradle}, iOS verwendet \textit{Xcode}. Die durch den Metro-Bundler gebündelten JavaScript-Dateien werden zu nativem Code kompiliert und anschließend in eine \textit{APK} für Android oder eine \textit{IPA} für iOS verpackt und signiert.

\vspace{0.5cm}

React Native bietet von Haus aus keine Möglichkeit, die Anwendung für das Web zu kompilieren. Hierfür kann ein zusätzliches Repository genutzt werden, das die Möglichkeit bietet, den React-Quellcode für den Browser zu kompilieren. Der Kompilierungsprozess ist hierbei identisch zu der Kompilierung von klassischem React-Code.

\vspace{0.5cm}

\begin{center}
    \url{https://reactnative.dev/docs/out-of-tree-platforms}
\end{center}

Ein Auszug der Möglichkeiten, für welche Endgeräte React Native kompiliert werden kann.

\section{Flutter im Detail}
Flutter ist ein von Google entwickeltes Cross-Platform-Framework. Es wurde erstmalig 2017 als Open-Source-Software veröffentlicht. Flutter nutzt eine eigene Architektur beim Aufbau der Software. Es werden Widgets genutzt, um Darstellung, Logik und Interaktionen innerhalb eines Objektes zu vereinen. Vorgefertigte Komponenten erleichtern den Entwicklungsprozess, indem oft genutzte Elemente, wie Buttons, Texte, Checkboxen usw., vordefiniert sind. Diese Komponenten orientieren sich in ihrem Design an den Richtlinien der jeweiligen Zielsysteme. Bei Android wird das Material-Design verwendet. Flutter nutzt als Programmiersprache \textit{Dart}, die entwickelt wurde, um eine Alternative zu JavaScript zu bieten.

\vspace{0.5cm}

Bei der Kompilierung ist das Ziel von Flutter, die bestmögliche Performance zu erzielen. Deshalb wird ein anderer Ansatz als bei React Native verwendet. Flutter kompiliert die Komponenten nicht zu nativen Komponenten, sondern zeichnet direkt jeden Pixel auf den Bildschirm des Endgeräts. Dadurch wird eine höhere Leistung erzielt und Flutter erhält große Kontrolle darüber, wie die Komponenten interagieren. Flutter verwendet \textit{Dart}, \textit{C} und \textit{C++}, die innerhalb der Dart Virtual Machine und der Grafik-Rendering-Engine genutzt werden.

\vspace{0.5cm}

Der kompilierte Code für die Grafik-Rendering-Engine wird bei Android mit dem NDK (\textit{Native Development Kit}) und bei iOS mit LLVM kompiliert. Anschließend wird die Dart VM mit \textit{Just-in-Time} (JIT) Kompilierung und eine VM mit \textit{Ahead-of-Time} (AOT) Kompilierung verwendet, um den Code in Maschinencode zu übersetzen. Die JIT-Kompilierung ermöglicht das Feature \textit{Hot Reload}, bei dem Änderungen im Quellcode kompiliert werden und in sehr kurzer Zeit in der Entwicklungsumgebung sichtbar sind. Die AOT-Kompilierung erstellt eine APK oder IPA, welche für die Veröffentlichung und Installation auf den jeweiligen Endgeräten genutzt werden kann.

\vspace{0.5cm}

\textit{Dart Web} ermöglicht die Kompilierung für das Web. Der Dart-Quellcode wird wahlweise in JavaScript oder WebAssembly übersetzt, wodurch er im Browser ausgeführt werden kann.

\vspace{0.5cm}

Die \textit{Dart SDK} stellt die nötigen Command-line-Tools zur Verfügung, um die Entwicklung zu erleichtern. Dazu gehört der \textit{Linter}, der \textit{Dart pub Package Manager} und viele weitere Funktionen.

\vspace{0.5cm}

Die nativen Funktionen des Endgeräts in Flutter werden über \textit{Method Channels} aufgerufen. Flutter selbst ist plattformunabhängig und bietet daher keinen direkten Zugriff auf system- oder gerätespezifische APIs. Um dennoch auf native Funktionen (wie z.B. die Kamera oder Galerie) zugreifen zu können, wird ein sogenannter \textit{Method Channel} verwendet.

\vspace{0.5cm}

Ein \textit{Method Channel} ermöglicht die Kommunikation zwischen dem Dart-Code von Flutter und dem nativen Code der jeweiligen Plattform (Android oder iOS). Dazu muss ein Channel erstellt werden, der als Brücke zwischen dem Flutter-Frontend und den API-Funktionen des Endgeräts fungiert. Der Dart-Code sendet Anfragen über diesen Kanal und der native Code antwortet, indem er die entsprechenden Betriebssystemfunktionen aufruft.

\vspace{0.5cm}

Beispielsweise wird bei der Auswahl eines Bildes über den \textit{Method Channel} eine Anfrage an das Endgerät gesendet, die die native Funktion zum Öffnen der Galerie aufruft. Nach der Auswahl eines Bildes wird dieses zurückgesendet und kann im Flutter-Quellcode verwaltet und angezeigt werden.


\section{Anforderungsanalyse}
In der Diskussion interpretierst du die Ergebnisse und stellst sie in den Kontext der bestehenden Forschung.

\section{Implementierung}
In diesem Abschnitt beschreibst du die Implementierung deiner Lösung und deren Funktionsweise.

\section{Ergebnisse}
Hier werden die Ergebnisse deiner Implementierung detailliert dargestellt.

\section{Diskussion}
In der Diskussion werden die Ergebnisse interpretiert und mit der bestehenden Forschung in Beziehung gesetzt.

\section{Fazit}
Das Fazit fasst die wichtigsten Erkenntnisse deiner Arbeit zusammen und gibt einen Ausblick auf mögliche zukünftige Forschungsfragen.

\newpage

\printbibliography

\appendix
\section{Anhang}
Im Anhang kannst du zusätzliche Informationen, wie z.B. Rohdaten, Fragebögen oder ausführliche Berechnungen, anführen.

\end{document}