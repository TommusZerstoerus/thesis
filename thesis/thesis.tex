\documentclass[a4paper,12pt]{article}
\usepackage[ngerman]{babel}
\usepackage[T1]{fontenc}
\usepackage[utf8]{inputenc}
\usepackage{lmodern}
\usepackage[margin=3cm]{geometry}
\usepackage{setspace}
\usepackage{amsmath, amssymb}
\usepackage{graphicx}
\usepackage{caption}
\usepackage{subcaption}
\usepackage[colorlinks=true, allcolors=gray]{hyperref}
\usepackage{csquotes}
\usepackage{biblatex}
\usepackage{fancyhdr}
\usepackage{tocbibind}

\title{Bachelorarbeit}
\author{Tom Becke}
\date{\today}

\begin{document}

\doublespacing

\maketitle

\begin{center}
    {\LARGE \textbf{Leistungs- und Renderzeitvergleich von React Native und Flutter: Eine Analyse der Performance moderner Cross-Plattform-Frameworks}}
\end{center}

\begin{abstract}
    Abstract kommt hier irgendwann hin.
\end{abstract}

\newpage

\tableofcontents

\newpage

\section{Einführung}
Heutzutage gibt es verschiedene Methoden, um Software effizient zu entwickeln. Um eine gelungene Oberfläche für den Benutzer zu gestalten, gibt es verschiedene Ansätze. Von der nativen Entwicklung, die explizit für das jeweilige Gerät optimiert ist, über die Cross-Platform Entwicklung, die einen Code verwendet um diesen auf verschiedenen Geräten abzubilden, bis hin zur webbasierten Entwicklung, die universell über einen Browser auf allen Geräten funktioniert, gibt es für den Entwickler diverse Möglichkeiten. Dabei sind viele Kriterien von großer Bedeutung. Dazu zählt die Auslastung auf den Endgeräten, da diese das Erlebnis, welches der Nutzer mit der Software hat, stark beeinflusst. Hinzu kommt die Performance, wie schnell die gewünschten Komponenten der Software angezeigt werden und in welcher Bildwiederholrate dies geschieht. Diese Faktoren werden in dieser Bachelorarbeit näher beleuchtet und in einer Studie verglichen.

Cross-Platform Frameworks haben in letzter Zeit viel Aufmerksamkeit erhalten, da sie in der Praxis wenig zeitintensiv in der Handhabung sind und bekannte Methoden durch verschiedene Techniken auf unterschiedliche Geräte projizieren können.

Die am häufigsten verwendeten Cross-Plattform Frameworks Flutter und React Native stehen im Mittelpunkt dieser Arbeit. Diese beiden Frameworks werden hinsichtlich der Auslastung auf den Endgeräten sowie der Performance beim Rendern verschiedener Komponenten verglichen. Die Forschungsfrage entstand vor dem Hintergrund, welches der beiden Frameworks in der Praxis für einen Großteil der Endgeräte in Bezug auf die genannten Aspekte optimaler ist. In einem Versuchsaufbau mit identischen Voraussetzungen werden die Frameworks auf verschiedene Punkte getestet und die Ergebnisse verglichen. Die abschließende Diskussion gibt Aufschluss darüber, welches der beiden Frameworks unter Berücksichtigung der genannten Punkte in der Praxis besser geeignet ist.

\section{Methoden der Softwareentwicklung}
Hier erläuterst du die theoretischen Grundlagen, die für das Verständnis deiner Arbeit notwendig sind.

\section{React Native im Detail}
In diesem Abschnitt beschreibst du die Methoden, die du verwendet hast, um deine Forschungsfrage zu beantworten.

\section{Flutter im Detail}
Hier präsentierst du die Ergebnisse deiner Forschung. Verwende Grafiken und Tabellen, um die Daten zu veranschaulichen.

\section{Anforderungsanalyse}
In der Diskussion interpretierst du die Ergebnisse und stellst sie in den Kontext der bestehenden Forschung.

\section{Implementierung}
In diesem Abschnitt beschreibst du die Implementierung deiner Lösung und deren Funktionsweise.

\section{Ergebnisse}
Hier werden die Ergebnisse deiner Implementierung detailliert dargestellt.

\section{Diskussion}
In der Diskussion werden die Ergebnisse interpretiert und mit der bestehenden Forschung in Beziehung gesetzt.

\section{Fazit}
Das Fazit fasst die wichtigsten Erkenntnisse deiner Arbeit zusammen und gibt einen Ausblick auf mögliche zukünftige Forschungsfragen.

\newpage

\printbibliography

\appendix
\section{Anhang}
Im Anhang kannst du zusätzliche Informationen, wie z.B. Rohdaten, Fragebögen oder ausführliche Berechnungen, anführen.

\end{document}