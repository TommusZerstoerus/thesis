,\section{Fazit}
Die Ergebnisse dieser Arbeit zeigen, dass sowohl Flutter als auch React Native ihre spezifischen Stärken und Schwächen haben, die von der jeweiligen Anwendung und den Einsatzbedingungen abhängen. Flutter zeichnet sich in den meisten Benchmarks durch eine bessere Performance aus, insbesondere bei grafikintensiven Aufgaben und der Animation einer großen Anzahl von Elementen. Dies liegt vor allem an der nativen Rendering-Engine Skia, die eine direkte Kontrolle über die Darstellung der UI bietet. React Native hingegen ist durch die JavaScript-Bridge limitiert, die in komplexen Szenarien zu Leistungseinbußen führt.

Im Bereich des State Managements zeigt Flutter durch seine AOT-Kompilierung und das effiziente Widget-System eine stabilere Leistung, während React Native durch die Bridge-Architektur auf mobilen Geräten Schwierigkeiten hat. Beim Speichern und Verarbeiten von Daten hat Flutter ebenfalls Vorteile, da es auf plattformspezifische Optimierungen wie SharedPreferences setzt. React Native kann in einigen Szenarien im Web konkurrenzfähig bleiben, jedoch treten auf mobilen Plattformen deutliche Schwächen auf.

Die Untersuchung hat zudem gezeigt, dass die Wahl der Frameworks stark von den spezifischen Anforderungen eines Projekts abhängt. Während Flutter durch seine hohe Performance und plattformübergreifende Konsistenz punktet, bietet React Native eine größere Flexibilität bei der Integration nativer Funktionen und eine stärkere Anpassung an das Design der Zielplattformen.

Zukünftige Entwicklungen könnten diese Unterschiede weiter beeinflussen. So könnten die neue Fabric-Architektur in React Native und die Integration von Skia als Rendering-Engine erhebliche Performance-Verbesserungen bringen. Auch Flutter bleibt durch kontinuierliche Optimierungen ein leistungsstarkes Framework für moderne Cross-Plattform-Anwendungen.

Abschließend lässt sich festhalten, dass Flutter derzeit bei den getesteten Benchmarks die bessere Performance bietet, während React Native durch seine Vielseitigkeit und größere Community besticht. Eine eindeutige Empfehlung, welches Framework in der täglichen Praxis zu empfehlen ist, kann jedoch nicht gegeben werden, da dies von vielen Faktoren abhängt. Dazu gehören die Erfahrung der Entwickler, die persönliche Präferenz und der Anwendungsfall. Die praxisnahen Messungen haben jedoch gezeigt, dass die Frameworks im Alltag eine nahezu identische Performance erzielen.